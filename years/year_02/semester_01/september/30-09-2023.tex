\documentclass{article}
\usepackage[utf8]{inputenc}

\usepackage[T2A]{fontenc}
\usepackage[utf8]{inputenc}
\usepackage[russian]{babel}

\usepackage{tabularx}
\usepackage{amsmath}
\usepackage{pgfplots}
\usepackage{geometry}
\geometry{
    left=1cm,right=1cm,top=2cm,bottom=2cm
}
\newcommand*\diff{\mathop{}\!\mathrm{d}}

\newtheorem{definition}{Определение}
\newtheorem{theorem}{Теорема}

\DeclareMathOperator{\sign}{sign}

\usepackage{hyperref}
\hypersetup{
    colorlinks, citecolor=black, filecolor=black, linkcolor=black, urlcolor=black
}

\title{Английский язык}
\author{Лисид Лаконский}
\date{September 2023}

\begin{document}
\raggedright

\maketitle

\tableofcontents
\pagebreak

\section{Практическое занятие — 30.09.2023}

\subsection{Exercise 185}

\begin{enumerate}
    \item Kolmogorov, Andrey Nikolaevich is a Russian mathematician whose work influenced several branches of modern mathematics, in particular he presented basic postulates for probability theory.  
    \item At the of 17 Kolmogorov was enrolled in the Moscow State University. In the autumn of 1921 he worked on complex problems in the theory of trigonometrical series and operations on sets. In the spring of 1922 he completed a study in the theory of operations on sets that was published in 1928.
    \item In 1925 Kolmogorov graduated from the faculty of physics and mathematics and was appointed a research associate of the faculty. He became interested in probability and his «General theory of measure and probability theory», that was later expanded into the monograph «Foundations of the Theory of Probability», gave the first description of an axiomatic construction of probability theory.
    \item In 1931 Kolmogorov was elected a professor of Moscow State University; two years later he was appointed a director of the Institute of Mathematics of The University. During the period of this appointment he elaborated the principles of stochastic processes, including the Markov processes, in the monograph «Analytical Methods Of Probability Theory». Other contributions concerned aspects of functional analysis, topology and turbulent flow of fluids. In the study of probability theory Kolmogorov formulated two systems of partial differential equations describing transition probabilities controlling a Markov process. The work marked a new period in the development of probability theory and its application in many fields.
\end{enumerate}

\subsection{Домашнее задание}

\begin{enumerate}
    \item Быть готовым пересказывать текст из Exercise 184 или Exercise 185 (до Миши Романчука — Колмогоров, после — Попов)
    
    \textbf{Exercise 184, конспект}: 
    
    \begin{enumerate}
        \item \textbf{Оглавление.} Alexander Popov is the 1st Inventor of the Radio.
        \item \textbf{Интерес и работа в тяжелых условиях.} It was about 100 years ago. At that time electrical engineering was a new science. Popov took great interest in it and began to work at it. His work went on under very hard conditions, but despite all the difficulties, Popov did not stop his experiments.
        \item \textbf{Демонстрация первых радиограмм. Разрешение на эксперименты.} On March 24, 1896, Popov made a report on the results of his work and demonstrated the world's first radiograms. After his report the government gave him permission to make experiments on board a small ship.
        \item \textbf{Хитрый бизнесмен Маркони, желающий всё украсть.} At this time an Italian, Marconi, began making the same kind of experiments. He already knew about Popov's experiments and plagiarized Popov's ideas to make money out of them. He wanted to show he was the 1st inventor of the radio.
        \item \textbf{Маркони уехал в Лондон и получил патент; хотел скрыть Попова.} In London he found the protection of rich businessman and the English government, so that in June, 1897, he received a patent for his invention and organized a commercial company. This company wanted to keep Popov in the background and gave ten thousand roubles to a Russian newspaper which often printed articles about Marconi's work.
        \item \textbf{Попов был ученым, а Маркони крысой поганой.} Popov and Marconi were people of a different kind. Popov was a great scientist, whereas Marconi was a young businessman.
        \item \textbf{Пока Попов бомжевал, Маркони ел устрицы.} While the great Russian inventor lived and worked under very hard conditions, Marconi's company already had a capital of 2 millions rubles.
        \item \textbf{Вот наконец Попов построил первую в мире радиостанцию.} It was only in 1899 that Popov could build a radio station. This was the 1st radio station in the world.
        \item \textbf{На первой международной конференции Маркони хотел всех обмануть, но французский ученый вмешался.} In August 1903 Popov took part in the work of the 1st International Conference on the Wireless. Marconi was also present at this conference, and tried to prove that he was the inventor of the radio. But the French scientist Bordelongue protested against this and spoke about Popov and his great invention.
        \item \textbf{Благодаря русско-японской войне правительство наконец поняло, что радио это круто, но было уже поздно.} In 1904, during the Russian-Japanese War, the government realized the importance of the radio as the only means of long-distance communication between ships. Then the government gave money and was ready to supply Popov with all the necessary equipment, but there were no instruments and no specialists in Russia, so it was quite impossible to do anything.
        \item \textbf{Смерть.} Popov died in 1905.
        \item \textbf{Признание первым изобретателем, которого не было.} A few years after Popov's death, the Russian Physical Society set up a commission to settle the question of Popov's invention. This commission stated that Popov was the 1st inventor of the radio.
    \end{enumerate}
    \item Theory of trigonometrical series and operations on sets

    Trigonometrical series is the expansion of trigonometric functions into a series.
    
    About operations on set, you already know all of them: it's union of sets, intersection of sets, difference of sets.
\end{enumerate}

\end{document}