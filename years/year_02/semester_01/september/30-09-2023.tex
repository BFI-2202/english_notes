\documentclass{article}
\usepackage[utf8]{inputenc}

\usepackage[T2A]{fontenc}
\usepackage[utf8]{inputenc}
\usepackage[russian]{babel}

\usepackage{tabularx}
\usepackage{amsmath}
\usepackage{pgfplots}
\usepackage{geometry}
\geometry{
    left=1cm,right=1cm,top=2cm,bottom=2cm
}
\newcommand*\diff{\mathop{}\!\mathrm{d}}

\newtheorem{definition}{Определение}
\newtheorem{theorem}{Теорема}

\DeclareMathOperator{\sign}{sign}

\usepackage{hyperref}
\hypersetup{
    colorlinks, citecolor=black, filecolor=black, linkcolor=black, urlcolor=black
}

\title{Английский язык}
\author{Лисид Лаконский}
\date{September 2023}

\begin{document}
\raggedright

\maketitle

\tableofcontents
\pagebreak

\section{Практическое занятие — 30.09.2023}

\subsection{Exercise 185}

\begin{enumerate}
    \item Kolmogorov, Andrey Nikolaevich is a Russian mathematician whose work influenced several branches of modern mathematics, in particular he presented basic postulates for probability theory.  
    \item At the of 17 Kolmogorov was enrolled in the Moscow State University. In the autumn of 1921 he worked on complex problems in the theory of trigonometrical series and operations on sets. In the spring of 1922 he completed a study in the theory of operations on sets that was published in 1928.
    \item In 1925 Kolmogorov graduated from the faculty of physics and mathematics and was appointed a research associate of the faculty. He became interested in probability and his «General theory of measure and probability theory», that was later expanded into the monograph «Foundations of the Theory of Probability», gave the first description of an axiomatic construction of probability theory.
    \item In 1931 Kolmogorov was elected a professor of Moscow State University; two years later he was appointed a director of the Institute of Mathematics of The University. During the period of this appointment he elaborated the principles of stochastic processes, including the Markov processes, in the monograph «Analytical Methods Of Probability Theory». Other contributions concerned aspects of functional analysis, topology and turbulent flow of fluids. In the study of probability theory Kolmogorov formulated two systems of partial differential equations describing transition probabilities controlling a Markov process. The work marked a new period in the development of probability theory and its application in many fields.
\end{enumerate}

\subsection{Домашнее задание}

\begin{enumerate}
    \item Быть готовым пересказывать текст из Exercise 184 или Exercise 185 (до Миши Романчука — Колмогоров, после — Попов)
    \item Theory of trigonometrical series and operations on sets

    Trigonometrical series is the expansion of trigonometric functions into a series.
    
    About operations on set, you already know all of them: it's union of sets, intersection of sets, difference of sets.
\end{enumerate}

\end{document}