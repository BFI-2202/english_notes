\documentclass{article}
\usepackage[utf8]{inputenc}

\usepackage[T2A]{fontenc}
\usepackage[utf8]{inputenc}
\usepackage[russian]{babel}

\usepackage{amsmath}

\title{Английский язык}
\author{Лисид Лаконский}
\date{September 2022}

\begin{document}

\maketitle
\tableofcontents
\pagebreak

\section{Английский язык — неизвестная дата}

В отрицательных формах частица not всегда ставится после первого вспомогательного глагола. В вопросах первый вспомогательный глагол переносится на первую позицию.

\subsection{Simple tenses}

\textbf{1. Регулярные повторяющиеся действия}: always, often, sometimes, rarely/seldom, never, every ..., once a ..., twice, three times a ..., every other day, on Monday(s)

\subsubsection{Present simple}

Положительная форма: \_ $V_{1}/V_{s/es}$

\noindent Отрицательная форма: \_ do/does not $V_{1}$

\noindent Вопросительная форма: do/does \_ $V_{1}$?

\hfill

\noindent Когда используется:

\indent \textbf{1. Состояние:} to live, to be, to want, to like, to have got, to see, to hear, to feel, to think, to smell, to taste...

\indent \textbf{2. Будущее расписание}

\subsubsection{Past simple}

Положительная форма: \_ $V_{2}/V_{ed}$

\noindent Отрицательная форма: \_ did not $V_{1}$

\noindent Вопросительная форма: did \_ $V_{1}$?

\hfill

\noindent Когда используется:

\indent \textbf{1. Состояния кроме to live}

\indent \textbf{2. Последовательные действия в прошлом}

\indent \textbf{3. Действия, которые совершились в законченный отрезок времени:} yesterday, last ..., ago ..., in 2010

\indent \textbf{4. Вопросы с when did}

\subsubsection{Future simple}

Положительная форма: \_ will $V_{1}$

\hfill

\noindent Когда используется:

\indent \textbf{1. Состояния кроме to live}

\indent \textbf{2. Прогнозы (предсказания) на будущее:} soon, perhaps, in the near future

\indent \textbf{3. Спонтанные действия в будущем}


\end{document}
