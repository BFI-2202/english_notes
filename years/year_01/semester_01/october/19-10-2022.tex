\documentclass{article}
\usepackage[utf8]{inputenc}

\usepackage[T2A]{fontenc}
\usepackage[utf8]{inputenc}
\usepackage[russian]{babel}

\usepackage{amsmath}
\usepackage{parcolumns}

\title{Английский язык}
\author{Лисид Лаконский}
\date{October 2022}

\begin{document}

\maketitle
\tableofcontents
\pagebreak

\section{Английский язык — 19.10.2022}

\subsection{Perfect tenses}

\subsubsection{Present perfect}

\begin{flushleft}

Действие совершалось в прошлом, но мы говорим о результате в настоящем.

\hfill

(+) \_ $\frac{\text{have}}{\text{has}}$ $V_{3/\text{ed}}$

\hfill

Подсказки:

\begin{enumerate}
    \item already, just, yet, so far, before, recently, ever, never, recently, lately
    \item Действие совершилось в незаконченный отрезок времени: today, this week, this month, this year, this <period of time>
    \item It's the $1_{\text{st}}$/$2_{\text{nd}}$/$100_{\text{th}}$ time when I have $V_3$
\end{enumerate}

\subsubsection{Past perfect}

(+) \_ had $V_3$

\subsubsection{Future perfect}

(+) \_ will have $V_3$

\pagebreak
\subsection{Домашняя работа}

Упражнение 71 - пересказ в любом времени.

\end{flushleft}

\end{document}