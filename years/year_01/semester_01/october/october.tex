\documentclass{article}
\usepackage[utf8]{inputenc}

\usepackage[T2A]{fontenc}
\usepackage[utf8]{inputenc}
\usepackage[russian]{babel}

\usepackage{amsmath}

\title{Английский язык}
\author{Лисид Лаконский}
\date{October 2022}

\begin{document}

\maketitle
\tableofcontents
\pagebreak

\section{Английский язык — неизвестная дата}

\subsection{Continuous tenses}

\subsubsection{Present continuous}

\noindent Когда используется:

\indent 1. Действия, происходящие в момент речи

\indent 2. Временные ситуации

\indent 3. Будущие запланированные действия (поезда, расписания и так далее)

\indent 4. Раздражающие чужие привычки: always, constantly

\subsubsection{Past continuous}

\noindent Когда используется:

\indent 1. Действия, происходящие в определенный момент в прошлом: at 5 pm yesterday

\indent 2. Продолжительные действия, «прерванные» короткими действиями

\indent 3. Одновременные продолжительные действия в прошлом


\subsubsection{Future continuous}

\noindent Когда используется:

\indent 1. Действия, которые будут происходить продолжительно

\indent 2. Продолжительные действия, которые будут «прерваны» короткими действиями

\indent 3. Одновременные продолжительные действия в будущем


\end{document}
