\documentclass{article}
\usepackage[utf8]{inputenc}

\usepackage[T2A]{fontenc}
\usepackage[utf8]{inputenc}
\usepackage[russian]{babel}

\usepackage{amsmath}
\usepackage{multienum}
\usepackage{geometry}

\geometry{
	left=1cm, right=1cm,
	top=1cm, bottom=2cm
}

\usepackage{hyperref}
\hypersetup{
    colorlinks,
    citecolor=black,
    filecolor=black,
    linkcolor=black,
    urlcolor=black
}

\title{Английский язык}
\author{Лисид Лаконский}
\date{April 2023}

\begin{document}
\raggedright

\maketitle
\tableofcontents
\pagebreak

\section{Perfect Tenses}

\subsection{What are the Perfect Tenses}

The perfect Tenses all include a form of the \textbf{auxiliary HAVE and the past participle} (V3). The progressive tenses always have the \textbf{past participle of BE (been) and the main verb in the -ING form}.

Examples:

\begin{multienumerate}
	\mitemxx{\textbf{present perfect}:  subject + has/have + past participle (+ object)}{\textbf{present perfect continuous}: subject + has/have + been + verb-ING  (+ object)}
	\mitemxx{\textbf{past perfect}: subject + had + past participle (+ object)}{\textbf{past perfect continuous}: subject + had + been + verb-ING  (+ object)}
	\mitemxx{\textbf{future perfect}: subject + will + have + past participle (+ object)}{\textbf{future perfect continuous}: subject + will + have + been + verb-ING  (+ object)}
\end{multienumerate}

\subsection{Connections in Time}

\begin{enumerate}
    \item the present perfect \textbf{connects the past to the present} (and \textbf{shows experience})
    \item the past perfect \textbf{connects 2 past actions} (or a past action and a past time), and \textbf{shows which happened first}
    \item the future perfect \textbf{connects a past, present, or future action with a time (or action) in the future}, and it \textbf{shows which will happen before that future time}
\end{enumerate}

The main \textbf{difference between the simple and continuous tenses} is that the \textbf{simple perfect tenses focus on a result} (something finishing), and the \textbf{continuous tenses focus on duration} (something continuing).

\subsection{Present Perfect \& Present Perfect Continuous}

I started teaching in the past, in 2007. Because today the year is 2017, I can connect the past to the present by saying:

\begin{multienumerate}
	\mitemxx{I have been teaching for 10 years.}{I have been a teacher since 2007.}
\end{multienumerate}

Remember, the \textbf{present perfect continuous focuses on duration, or something continuing}. With this tense, I am emphasizing that continuing action (rather than a finished result). In my second example, the main verb is BE, which is a non-action or stative verb. Non-action verbs can never be in the -ING form, so that’s why it’s in the [simple] present perfect tense. \textbf{We use the [simple] present perfect to show continuation with stative verbs only}.

\hfill

The \textbf{simple present perfect is usually only used to show experience} though. For example, I have lived in Brazil. This means I have the experience of living in Brazil.

\subsection{Past Perfect \& Past Perfect Continuous}

I started studying in college in 2004. This action happened before another past action of starting to teach in 2007. So, I can connect two past events by saying:

\begin{multienumerate}
	\mitemxx{I had been studying for 3 years when I started teaching.}{I had studied in college by the time I started teaching.}
\end{multienumerate}

Both of the actions in \textbf{the past perfect tenses} (had been studying \& had studied) show that they happened 1st, BEFORE the 2nd action of starting to teach. Remember that the main difference between these two examples is that the \textbf{continuous tense focuses on continuation/duration}, and the \textbf{simple tense focuses on an action finishing/the result}.

As I mentioned we can connect 2 past actions like my examples above (studying \& teaching), or \textbf{we can connect a past action with a past time}:

\begin{multienumerate}
	\mitemxx{I had been studying for 3 years by 2007}{I had studied in college by 2007}
\end{multienumerate}

\subsection{Future Perfect \& Future Perfect Continuous}

I started teaching in 2007. If I wanted to connect this past action, to the future (the year 2021), then I can do this by saying:

\begin{enumerate}
	\item I will have been teaching for 14 years by 2021.
	\item *I will have graduated college by 2021. — (( This is an example of the [simple] future perfect \textbf{showing an action will have been finished by a future time, focusing on the result}.))
\end{enumerate}

Let’s imagine that I bought a house today (in 2017). That would make it a present tense action. I can connect the present to the future by saying:

\begin{enumerate}
	\item I will have owned a house for 4 years by 2021. — (( This is an example of a stative verb (OWN) showing \textbf{continuation with the simple future perfect}.))
	\item I will have been living in my new (owned) home for 4 years by 2021. — (( This \textbf{focuses on continuation with the future perfect continuous}.))
\end{enumerate}

Finally, we can connect the future to the future. So I know (it’s a pretty for sure plan) that I will have a second baby before 2021. I can express this by saying:

\begin{enumerate}
	\item I will have had a second baby by 2021. — (( This \textbf{focuses on a result happening before a future event} ))
\end{enumerate}

\pagebreak
\section{The Passive Voice}

We use the passive voice to change the focus of the sentence.

We often use the passive:

\begin{enumerate}
	\item when we prefer not to mention who or what does the action (for example, it's not known, it's obvious or we don't want to say)
	\item so that we can start a sentence with the most important or most logical information
	\item in more formal or scientific writing.
\end{enumerate}

\subsection{How do we make the passive}

\textbf{We make the passive using the verb be + past participle. We start the sentence with the object.
}

\begin{center}
\begin{tabular}{||c c c||} 
 \hline
 Tense & Active & Passive \\ [0.5ex] 
 \hline\hline
 present simple & I \textbf{make} a cake. & A cake \textbf{is made} (by me).  \\ 
 \hline
 present continuous & I \textbf{am making} a cake. & A cake \textbf{is being made} (by me).  \\
 \hline
 past simple & I \textbf{made} a cake. & A cake \textbf{was made} (by me).  \\
 \hline
 past continuous & I \textbf{was making} a cake. & A cake \textbf{was being made} (by me).  \\
 \hline
 present perfect & I \textbf{have made} a cake. & A cake \textbf{has been made} (by me).  \\
 \hline
 pres. perf. continuous & I \textbf{have been making} a cake. & A cake \textbf{has been being made} (by me).  \\
 \hline
 past perfect & I \textbf{had made} a cake. & A cake \textbf{had been made} (by me).  \\
 \hline
 future simple & I \textbf{will make} a cake. & A cake \textbf{will be made} (by me).  \\
 \hline
 future perfect & I \textbf{will have made} a cake. & A cake \textbf{will have been made} (by me).  \\ [1ex] 
 \hline
\end{tabular}
\end{center}

\pagebreak
\section{Conditionals}

Conditionals \textbf{describe the result of a certain condition}. The \textbf{if clause tells you the condition} (If you study hard) and the \textbf{main clause tells you the result} (you will pass your exams). The order of the clauses does not change the meaning.

\subsection{Zero conditional}

We use the zero conditional to talk about \textbf{things that are generally true, especially for laws and rules}.

\begin{enumerate}
	\item If I drink too much coffee, I can't sleep at night.
	\item Ice melts if you heat it.
	\item When the sun goes down, it gets dark.
\end{enumerate}

The \textbf{structure} is: if/when + present simple >> present simple.

\subsection{First conditional}

We use the first conditional when we talk about \textbf{future situations we believe are real or possible}.

\begin{enumerate}
	\item If it doesn't rain tomorrow, we'll go to the beach.
	\item Arsenal will be top of the league if they win.
	\item When I finish work, I'll call you.
\end{enumerate}

In first conditional sentences, the \textbf{structure} is usually: if/when + present simple >> will + infinitive. 

It is also common to use this structure with \textbf{unless}, \textbf{as long as}, \textbf{as soon as} or \textbf{in case} instead of if.

\begin{enumerate}
	\item I'll leave as soon as the babysitter arrives.
	\item I don't want to stay in London unless I get a well-paid job.
	\item I'll give you a key in case I'm not at home.
	\item You can go to the party, as long as you're back by midnight.
\end{enumerate}

\subsection{Second conditional}

The second conditional is used to \textbf{imagine present or future situations that are impossible or unlikely in reality}.

\begin{enumerate}
	\item If we had a garden, we could have a cat.
	\item If I won a lot of money, I'd buy a big house in the country.
	\item I wouldn't worry if I were you.
\end{enumerate}

The \textbf{structure} is usually: if + past simple >> + would + infinitive. 

\textbf{When if is followed by the verb be}, \textbf{it is grammatically correct} to say \textbf{if I were}, \textbf{if he were}, \textbf{if she were} and \textbf{if it were}.

\subsection{Third conditional}

The third conditional is used to \textbf{imagine a different past}. We \textbf{imagine a change in a past situation and the different result of that change}.

\begin{enumerate}
	\item If I had understood the instructions properly, I would have passed the exam.
	\item We wouldn't have got lost if my phone hadn't run out of battery.
\end{enumerate}

In third conditional sentences, the \textbf{structure} is usually: If + past perfect >> would have + past participle.

\subsection{Mixed conditionals}

We can use mixed conditionals when we \textbf{imagine a past change with a result in the present} or a \textbf{present change with a result in the past}.

\begin{enumerate}
	\item \textbf{Past/Present} \\
	Here's a sentence imagining \textbf{how a change in a past situation would have a result in the present}: \\
	\ \ \ If I hadn't got the job in Tokyo, I wouldn't be with my current partner. \\
	So the \textbf{structure} is: If + past perfect >> would + infinitive.
	\item \textbf{Present/Past} \\
	Here's a sentence imagining \textbf{how a different situation in the present would mean that the past was different as well}: \\
	\ \ \  It's really important. If it wasn't, I wouldn't have called you on your holiday. \\
	And the \textbf{structure} is: If + past simple >> would have + past participle.
\end{enumerate}

\section{Модальные глаголы в английском языке}

Модальные глаголы в английском языке — это \textbf{глаголы, которые не имеют собственного значения, они могут использоваться только в связке с другим глаголом}. Английские модальные глаголы выражают модальность, то есть \textbf{отношение говорящего к какому-либо действию}. 

\subsection{Модальный глагол \textbf{can}}

Can в значении «мочь», «уметь» используется, чтобы выражать \textbf{возможность совершения действия}.

Две формы глагола can:

\begin{enumerate}
	\item can — настоящее время 
	\item could — прошедшее время и сослагательное наклонение 
\end{enumerate}

\subsubsection{Случаи употребления глагола can}

\begin{enumerate}
	\item \textbf{Умственная или физическая активность}.
	\begin{enumerate}
		\item Due to my spine problems, I can’t stand so long. — Из-за моих проблем с позвоночником я не могу стоять так долго (то есть, физически не могу).
		\item He could run faster. — Он мог бежать быстрее (физическая активность — бег).
		\item I can memorize 20 words in 5 minutes. — Я могу запомнить 20 слов за 5 минут (подразумевается умственная активность).
	\end{enumerate}
	 \item \textbf{Общая или теоретическая вероятность совершения действия}.
	 \begin{enumerate}
	 	\item She can do anything. — Она может cделать что угодно.
	 	\item You can get knowledge from books. — Вы можете получать знания из книг.
	 \end{enumerate}	 
	 \item \textbf{Выражение просьбы}. В данном случае можно употребить как can, так и could, но последний вариант будет более вежливым и формальным.
	 \begin{enumerate}
	 	\item Can you wait for me outside? — Можешь подождать меня на улице? 
	 \end{enumerate}
	 \item \textbf{Разрешение сделать что-либо, просьба о разрешении произвести определенные действия, запрет}
	 \begin{enumerate}
	 	\item Can I take a photo? — Могу я сделать фотографию? 
	 	\item You can do whatever you want. — Ты можешь делать все, что захочешь. 
	 	\item You cannot enter the room without my permission. — Вы не можете входить в комнату без моего разрешения.
	 \end{enumerate}
	 \item \textbf{Выражение удивления, упрека или недоверия}
	 \begin{enumerate}
	 	\item Can it be true? — Неужели это правда? 
	 	\item You could at least give me a hint! — Ты мог бы хотя бы намекнуть мне! 
	 	\item No, she can’t treat me like this. — Нет, она не может так поступать со мной. 
	 \end{enumerate}
\end{enumerate}


\subsection{Модальный глагол \textbf{be able to}}

Выражения с глаголом \emph{can} в \textbf{будущее время} переводятся с использованием глагола вероятности \emph{to be able to} (быть способным / в состоянии сделать). Он практически равнозначен глаголу \emph{can}, однако \textbf{в настоящем и прошедшем времени он чаще используется только для того, чтобы выразить, что лицу удалось что-то сделать, оно в чем-то преуспело}.

\begin{enumerate}
	\item \emph{I couldn’t speak Chinese but I was able to explain what I wanted}. — Я не мог говорить по-китайски, но я смог объяснить, чего я хочу.
	\item \emph{Carl will be able to move to England}. — Карл сможет переехать в Англию.
	\item \emph{She is able to participate in that play}. — Она может участвовать в этой пьесе.
\end{enumerate}

\subsection{Модальный глагол \textbf{may}}

К модальным глаголам, которые \textbf{выражают вероятность}, также относится глагол may в значении \textbf{«разрешите», «можно»}. 

Две формы глагола may:

\begin{enumerate}
	\item may — \textbf{настоящее время}
	\item might — \textbf{прошедшее время или сослагательное наклонение}
\end{enumerate}

\subsubsection{Случаи употребления глагола may}

\begin{enumerate}
	\item \textbf{Разрешение сделать что-либо}, просьба о разрешении произвести определенные действия
	\begin{enumerate}
		\item \emph{May I stay here?} — Можно мне остаться здесь? 
		\item \emph{You may stay}. — Вы можете остаться. 
	\end{enumerate}
	\item \textbf{Выражение неуверенности в вероятности осуществления действия}. Might может использоваться в роли усилителя:
	\begin{enumerate}
		\item \emph{I may apply to Harvard}. — Я, может быть, подам документы в Гарвард.
		\item \emph{It might be raining tonight}. — Может, вечером будет дождь. 
	\end{enumerate}
\end{enumerate}

\textbf{Для большей формальности} в обоих случаях можно использовать might.

\subsection{Модальный глагол \textbf{be allowed to}}

Аналогом модального глагола may является модальный глагол be allowed to \textbf{в значении разрешения}. Такой глагол используется, чтобы \textbf{показать, что разрешение было дано, без уточнения, кем это было сделано}. Так как глагол to be изменяется по временам, числам и лицам, эти же изменения применяются и для модального глагола be allowed to.

\begin{enumerate}
	\item \emph{George was allowed to buy some sweets}. — Джорджу разрешили купить немного сладостей.
	\item \emph{I’m allowed to edit the texts on the website}. — Мне разрешено редактировать тексты на веб-сайте.
	\item \emph{She is not allowed to enter}. — Ей не разрешено входить.
	\item \emph{He was not allowed to have a pet}. — Ему не разрешали иметь домашнее животное. 
\end{enumerate}

\subsection{Модальный глагол \textbf{must}}

Английский модальный глагол must \textbf{употребляется в значении «должен»}. Не имеет формы прошедшего и будущего времени. Может употребляться в сокращенной форме mustn’t. 

\subsubsection{Случаи употребления глагола must}

\begin{enumerate}
	\item \textbf{Выражение обязанности, необходимости}.
	\begin{enumerate}
		\item \emph{We must finish the essay before the deadline}. — Мы обязаны закончить эссе до крайнего срока сдачи. 
	\end{enumerate}
	\item \textbf{Вынужденное совершение действия}.
	\begin{enumerate}
		\item \emph{You must do this, or you will have problems}. — Вы должны сделать это, иначе у вас будут проблемы. 
	\end{enumerate}
	\item \textbf{Приказ или запрет}.
	\begin{enumerate}
		\item \emph{You must arrest him!} — Вы должны арестовать его!
		\item \emph{You must not break the rules}. — Вы не должны нарушать правила. 
	\end{enumerate}
	\item \textbf{Выражение уверенности}.
	\begin{enumerate}
		\item \emph{Charlie must be happy}. — Чарли должно быть (наверняка) счастлив. 
	\end{enumerate}
\end{enumerate}

\subsection{Модальный глагол \textbf{have to}}

Так как must не имеет \textbf{прошедшего и будущего времени}, вместо него в таких случаях используют модальный глагол have (has) to. Он изменяется по числам, лицам и временам. 

\begin{enumerate}
	\item \emph{We had to update the profiles}. — Нам нужно было обновить профили. 
	\item \emph{I’ll have to meet him}. — Я буду должен с ним встретиться. 
\end{enumerate}

Модальный глагол have to (have got to) используется в значении \textbf{«приходится», «должен»}. Помимо вышеупомянутого случая, он также употребляется и самостоятельно, не заменяя глагол must. \textbf{Обычно он используется, чтобы показать, что какое-то действие необходимо произвести «потому что надо, а не потому, что хочется»}.

\subsubsection{Различия между must и have to}

\begin{enumerate}
	\item Модальный глагол must в английском языке используется, когда есть \textbf{осознание обязанности, необходимости что-то сделать}, а также \textbf{существует правило, которому обязательно нужно следовать}.
	\begin{enumerate}
		\item \emph{You must help your parents}. — Ты должен помогать родителям. 
	\end{enumerate}
	\item При использовании have to, мы подразумеваем, что \textbf{мы не хотим что-то делать, но нам придется ввиду обстоятельств}.
	\begin{enumerate}
	 	\item \emph{We had to give in}. — Нам пришлось уступить. 
	 \end{enumerate} 
\end{enumerate}

\subsubsection{Различия между have to и have got to}

\begin{enumerate}
	\item Have got to \textbf{подразумевает конкретное действие}.
	\begin{enumerate}
		\item \emph{I’ve got to go to the dentist on Monday}. — Мне нужно пойти к стоматологу в понедельник. 
	\end{enumerate}
	\item Have to \textbf{подразумевает повторяющееся действие}.
	\begin{enumerate}
		\item \emph{I have to consult my doctor every time before I go abroad}. — Я должен консультироваться с врачом каждый раз перед тем, как ехать за границу. 
	\end{enumerate}
\end{enumerate}

\subsection{Модальный глагол \textbf{should}}

Модальный глагол should употребляется в значении \textbf{«должен», «следует»}. При отрицании имеет сокращенную форму shouldn’t.

\subsubsection{Случаи употребления глагола should}

\begin{enumerate}
	\item \textbf{Моральное обязательство}.
	\begin{enumerate}
		\item \emph{I should do something good for him}. — Я должен сделать для него что-то хорошее. 
		\item \emph{I should be proud of my child}. — Я должен гордиться своим ребёнком. 
	\end{enumerate}
	\item \textbf{Совет}.
	\begin{enumerate}
		\item \emph{You should avoid passive people}. — Тебе надо избегать пассивных людей. 
		\item \emph{She should learn foreign languages}. — Ей следует учить иностранные языки. 
		\item \emph{You should understand that there is nothing more important than your family}. — Вы должны понимать, что нет ничего более важного, чем ваша семья. 
	\end{enumerate}
	\item \textbf{Инструкции}.
	\begin{enumerate}
		\item \emph{You should mix the flour and the yeast}. — Вы должны смешать муку и дрожжи. 
	\end{enumerate}
\end{enumerate}

\subsection{Модальный глагол \textbf{ought to}}

Как и should, употребляется в значении \textbf{«должен», «следует»}, но используется намного реже. Этот модальный глагол имеет только одну форму. В отрицании может сокращаться до oughtn’t to. 

Ought to используется для выражения \textbf{советов и обязательств}. 

\begin{enumerate}
	\item \emph{We ought to complain about the quality}. — Мы должны пожаловаться на качество. 
	\item \emph{He ought to buy her flowers}. — Он должен купить ей цветы. 
	\item \emph{You ought to give all your love to children}. — Вы должны отдать всю свою любовь детям. 
\end{enumerate}

\subsection{Модальные глаголы \textbf{shall} и \textbf{will}}

Соединяют в себе \textbf{модальное значение и значение будущих времен}. В отрицании shall может сокращаться до shan’t, will — до won’t.

\textbf{Shall} используется в тех случаях, когда \textbf{требуется предложить сделать что-либо}.

\begin{enumerate}
 	\item \emph{Shall I open the door for you?} — Открыть для вас дверь? 
\end{enumerate} 

Английский модальный глагол \textbf{will} используется, если требуется \textbf{настоять на чем-либо}. Также его можно встретить в \textbf{вопросительных предложениях, которые подразумевают приказы}.

\begin{enumerate}
	\item \emph{You will clean the room}. — Тебе придется убрать комнату. 
	\item \emph{Will you keep quiet?} — Соблюдайте тишину! 
\end{enumerate}

\subsection{Модальный глагол \textbf{be to}}

Служит для \textbf{выражения обязательства}. Используется в прошедшем и настоящем временах. 

\subsubsection{Случаи употребления глагола be to}

\begin{enumerate}
	\item \textbf{Выражение действий, которые выполняются по определенному расписанию}.
	\begin{enumerate}
		\item \emph{The plane is to take off in 15 minutes}. — Самолет взлетает через 15 минут. 
	\end{enumerate}
	\item \textbf{Для действий, которые предопределены}.
	\begin{enumerate}
		\item \emph{The little girl was to become the most successful writer in the USA}. — Эта маленькая девочка станет в будущем самой успешной писательницей в США. 
	\end{enumerate}
	\item \textbf{Для того чтобы выразить запрет или невозможность}.
	\begin{enumerate}
		\item \emph{People are not to be in that area}. — Людям нельзя быть на этой территории.
		\item \emph{The language is not to be learnt in 6 months}. — Этот язык невозможно выучить за полгода. 
	\end{enumerate}
\end{enumerate}

\subsection{Модальный глагол \textbf{would}}

Используется для \textbf{вежливых просьб и предложений}. Следует различать would в значении «бы» и would как модальный глагол. Имеет сокращенную отрицательную форму wouldn’t.

Would употребляется в \textbf{фразах с предложениями, предположениями и просьбами}. 

\begin{enumerate}
	\item \emph{Would you turn on the computer, please?} — Не могли бы вы включить компьютер, пожалуйста?
	\item \emph{Would you like a pie, or a cake?} — Не хотите пирога или тортика? 
	\item \emph{It would be his brother over there}. — Наверно, это его брат вон там. 
\end{enumerate}

Этот модальный глагол может применяться в качестве аналога модального глагола used to, выражая \textbf{действия, которые происходили раньше, но уже не происходят сейчас}. 

\begin{enumerate}
	\item \emph{When I was little, I would watch cartoons every day}. — Когда я был маленьким, я смотрел мультфильмы каждый день. 
\end{enumerate}

\pagebreak
\section{Полумодальные глаголы в английском языке}

К полумодальным относят те глаголы, которые в высказывании \textbf{могут выступать и в роли основного глагола, и в роли модального}, в зависимости от смысла и конструкции предложения.

\subsection{Глагол \textbf{used to}}

\textbf{Употребляется только для выражения действий/состояний, случившихся в прошлом}. При переводе на русский у предложений с глаголом used to может появляться наречие «раньше». 

\begin{enumerate}
	\item \emph{I used to like skiing when I was young}. — Когда я был молодым, мне нравилось кататься на лыжах. 
	\item \emph{He didn’t use to (used not to) drink alcohol that much}. — Раньше он не употреблял столько алкоголя. 
	\item \emph{Max used to speak English fluently}. — Раньше Макс свободно владел английским языком. 
\end{enumerate}

Варианты образования \textbf{отрицания}: «didn’t use to» или «used not to».

Образование \textbf{вопроса}: «Did he use to … ?». 

\subsection{Глагол \textbf{need}}

Обозначает \textbf{надобность выполнения действия}. Сокращенная форма отрицания — needn’t.

\subsubsection{Случаи употребления глагола need}

\begin{enumerate}
	\item Передает значение «нужно» в утвердительных предложениях.
	\begin{enumerate}
		\item \emph{All we need is love}. — Все, что нам нужно — это любовь. 
		\item \emph{You needn’t do this until you are ready}. — Вам не нужно делать это, пока вы не будете готовы. 
	\end{enumerate}
	\item Используется в вопросах, когда в ответ автору хочется услышать отрицание.
	\begin{enumerate}
		\item \emph{Do I need to call her? I dislike that girl}. — Мне, правда, нужно ей позвонить? Мне так не нравится эта девчонка. 
	\end{enumerate}
\end{enumerate}

\subsection{Глагол \textbf{dare}}

Используется в значении \textbf{«рискнуть», «посметь что-то сделать»}. В качестве смыслового глагола также сохраняет эти значения. Не требует применения вспомогательных глаголов. 

\begin{enumerate}
	\item \emph{How dare you tell me what to do?!} — Как ты смеешь указывать, что мне делать? 
	\item \emph{They dare not ask about the salary}. — Они не осмеливаются спросить о зарплате. 
\end{enumerate}

\subsection{Глагол \textbf{let}}

Используется в значении \textbf{«пусть», «позвольте», «разрешите»}. В качестве смыслового глагола также сохраняет эти значения. 

\begin{enumerate}
	\item \emph{Let her think about it}. — Пусть она подумает об этом. 
	\item \emph{Let Michael not come tomorrow}. — Позвольте Майклу не приходить завтра. 
\end{enumerate}

\pagebreak
\section{Participles}

\subsection{Participle 1. Present Participle Active}

Present Participle Active можно легко узнать по окончанию \textbf{-ing}.

Итак, Present Participle Active соответствует \textbf{русскому действительному причастию настоящего времени}, и \textbf{деепричастию несовершенного вида}.

\begin{enumerate}
	\item \textbf{Причастные обороты, выражающие обстоятельство времени}
	\begin{enumerate}
		\item (When) going home I met an old friend of mine. - \textbf{с причастием} — Идя домой (когда я шел домой), я встретил своего старого друга.
		\item (While) having dinner we discussed many questions. - \textbf{с причастием} — Обедая (Пока мы обедали), мы обсудили много вопросов.
		\item When driving a car I always fasten the seat belt. - \textbf{с причастием} — Ведя машину (Когда я веду машину), я всегда пристегиваю ремень.
	\end{enumerate}
	\item \textbf{Обстоятельство образа действия / сопутствующего обстоятельства}
	\begin{enumerate}
		\item They (were drinking) drank beer, talking about women. - \textbf{с причастием} (сопутствующее обстоятельство) — Они пили пиво, разговаривая о женщинах.
		\item He (was reading) read the letter nervously laughing. - \textbf{с причастием} (обстоятельство образа действия). — Он (про)читал письмо нервно смеясь.
		\item 
	\end{enumerate}
	\item \textbf{Обстоятельство причины}
	\begin{enumerate}
		\item Earning much money he usually stayed in Hilton.- \textbf{с причастием} — Зарабатывая много денег, он остановился в Хилтоне.
	\end{enumerate}
\end{enumerate}

\subsection{Participle 1. Present Participle Passive}

Present Participle Passive образуется так: \textbf{being + 3 форма глагола}.

Present Participle Passive соответствует \textbf{русскому причастию в страдательном залоге}. Данная форма причастия \textbf{употребляется для выражения длительного действия, совершающегося в момент речи или в настоящий период времени}.

Как и Present Participle Active, данная форма причастия может выполнять в предложении функцию \textbf{определения} и \textbf{обстоятельства}.

\begin{enumerate}
	\item \textbf{Определение}
	\begin{enumerate}
		\item The test being written now is our final paper work — Тест, пишущийся сейчас - наша финальная работа
	\end{enumerate}
	\item Present Participle Passive в функции \textbf{обстоятельства} выражает значение \textbf{причины} и \textbf{времени}. В данном случае Present Participle Passive подчеркивает не одновременность действий в причастии и основном предложении, а \textbf{последовательность их совершения}. \\
	Но \textbf{действие, выраженное данным причастием, все-таки совершилось раньше действия, выраженного глаголом в основном предложении}. Present Participle Passive переводится на русский язык \textbf{страдательным деепричастием}. Также данная форма может переводиться и при помощи придаточного предложения
	\begin{enumerate}
		\item Being warned about the possible dangers we decided to take with us guns — Будучи предупрежденными о возможных опасностях, мы решили взять с собой ружья.
		\item Being sent to prison unjustly he remained calm — Будучи посланным в тюрьму незаслуженно, он оставался спокойным
	\end{enumerate}
\end{enumerate}

\subsection{Perfect Participle}

\subsubsection{Perfect Participle Active}

I Perfect Participle Active образуется по следующей формуле: \textbf{having + 3 форма глагола}.

Perfect Participle Active соответствует русскому \textbf{деепричастию совершенного вида}. Это причастие \textbf{выражает действие, предшествующее действию, выраженному глаголом в личной форме}.

В предложении Perfect Participle Active несет функции обстоятельства - \textbf{обстоятельства времени и причины}. Однако в некоторых случаях представляется трудным определить тип обстоятельства, так как оба значения могут быть очень близки. Если данное причастие заменить придаточным предложением, то сказуемое в нем будет иметь форму Past Perfect.

\begin{enumerate}
	\item \textbf{Обстоятельство времени}
	\begin{enumerate}
		\item \emph{Having finished his last book the famous writer decided to take a long holiday} — Закончив свою последнюю книгу, известный писатель решил отправиться на продолжительные каникулы.
		\item \emph{Coming home after a hard day, Robert had some beer and went to sleep} — Придя домой после тяжелого дня, Роберт выпил немного пива и пошел спать
	\end{enumerate}
	\item \textbf{Обстоятельство причины}
	\begin{enumerate}
		\item \emph{Having lost his wallet, he could not buy the present to his wife} — Потеряв свой бумажник, он не смог купить подарок своей жене
		\item \emph{He shut the book having read the necessary information} — Он захлопнул книгу, прочитав необходимую информацию
	\end{enumerate}
\end{enumerate}

\subsubsection{Perfect Participle Passive}

Perfect Participle Passive - вариант Perfect Participle в \textbf{страдательном залоге}, образуется следующим образом: \textbf{having + been + 3 форма глагола}

Как и Perfect Participle Active, его пассивная форма в предложении выполняет функции \textbf{обстоятельства времени и причины}

\begin{enumerate}
	\item \textbf{Обстоятельство времени}
	\begin{enumerate}
		\item \emph{Having been lost in the forest, the cat could find the way back home} — Будучи потерянной в лесу, кошка смогла найти обратную дорогу домой
	\end{enumerate}
	\item \textbf{Обстоятельство причины}
	\begin{enumerate}
		\item \emph{Having been cheated the woman went to the police} — Будучи обманутой, женщина пошла в полицию
		\item \emph{Being sorted the letters were sent to the addressees} — Будучи рассортированными, письма были отправлены адресатам.
	\end{enumerate}
\end{enumerate}

\subsection{Participle 2}

Причастие 2 иногда называют \textbf{причастием прошедшего времени} - Past Participle. У этого причастия всего одна неизменяемая форма, которая соответствует \textbf{3 форме глагола}. Причастие 2 \textbf{не имеет отдельных категорий времени и залога, так как может выражать действия, относящиеся к любому времени в зависимости от времени сказуемого}.

Форму Past Participle можно встретить во временах группы Perfect и в страдательном залоге.

Participle 2 \textbf{имеет свойства глагола, прилагательного (определения) и обстоятельства}.

\begin{enumerate}
	\item В функции \textbf{определения} Participle 2 может употребляться перед и после существительного:
	\begin{enumerate}
		\item \emph{There were many beautifully dressed people near the theatre} — Рядом с театром было много красиво одетых людей
		\item \emph{The documents lost in the park are of great importance} — Документы, потерянные в парке, очень важны.
	\end{enumerate}
	\item \textbf{Обстоятельство}
	\begin{enumerate}
		\item \textbf{Обстоятельство причины}
		\begin{enumerate}
			\item \emph{Captured by the traffic jam we missed the train} — Задержанные дорожной пробкой, мы опоздали на поезд.
		\end{enumerate}
		\item \textbf{Обстоятельство времени}
		\begin{enumerate}
			\item \emph{Taken by surprise he never showed his embarrassment} — Будучи застигнутым врасплох, он никогда не показывал своего смущения
			\item \emph{When asked stupid questions, he frowned and answered nothing} — Когда ему задавали глупые вопросы, он хмурился и ничего не отвечал
		\end{enumerate}
		\item \textbf{Обстоятельство условия}
		\begin{enumerate}
			\item \emph{If deleted these data will never be recovered} — Если эти данные будут стерты, их уже будет не восстановить.
		\end{enumerate}
		\item \textbf{Обстоятельство уступки}
		\begin{enumerate}
			\item \emph{Though scared the brave travelers continued their way} — Даже будучи напуганными, храбрые путешественники продолжили свой путь
		\end{enumerate}
	\end{enumerate}
\end{enumerate}

\end{document}