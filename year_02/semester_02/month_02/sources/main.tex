\documentclass{article}

\usepackage[T2A]{fontenc}
\usepackage[utf8]{inputenc}
\usepackage[russian]{babel}

\usepackage{tabularx}
\usepackage{amsmath}
\usepackage{pgfplots}
\usepackage{geometry}
\usepackage{multicol}
\geometry{
    left=1cm,right=1cm,top=2cm,bottom=2cm
}
\newcommand*\diff{\mathop{}\!\mathrm{d}}

\newtheorem{definition}{Определение}
\newtheorem{theorem}{Теорема}

\DeclareMathOperator{\sign}{sign}

\usepackage{hyperref}
\hypersetup{
    colorlinks, citecolor=black, filecolor=black, linkcolor=black, urlcolor=black
}

\title{Иностранный язык}
\author{Лисид Лаконский}
\date{Февраль 2023}

\begin{document}
\raggedright

\maketitle

\tableofcontents
\pagebreak

\section{Практическое занятие — 03.02.2024}

\subsection{Введение}

Экзаменационный билет будет состоять из следующего:

\begin{enumerate}
    \item Чтение текста профессионально-ориентированной тематики (1 час на ознакомление; можно пользоваться \textbf{печатным} словарем). Контроль — \textbf{устное собеседование}.
    \item Составление резюме по прочитанному тексту в соответствие с образцом (с которым мы уже знакомы; 1 час). \textbf{Рассказывать устно}. Записывать его не надо (только если очень хочется).
    \item Устно-речевое монологическое высказывание в рамках пройденных тем. Объем — 25 развернутых фраз. Список возможных тем:
    \begin{enumerate}
        \item История Российской Академии Наук
        \item Российская Академия Наук в наши дни
        \item МТУСИ (в общем). Кроме того, возможны две связанные темы:
        \begin{enumerate}
            \item О нашей учебе в МТУСИ — О нашем факультете, о нашей специальности... О планах на будущее (перспективах).
            \item Мой факультет — Можно почерпнуть с \url{https://mtuci.ru}.
        \end{enumerate}
        \item Конференции, симпозиумы, конгрессы (конечно, научные). Связанные темы:
        \begin{enumerate}
            \item Мой опыт участия в конференции.
        \end{enumerate}
        \item Глобальные проблемы окружающей среды (global environmental problems). Будет базовый текст, к нему надо придумать еще что-нибудь свое. Например: мне вот кажется, что главная проблема — то, что пингвины умирают на Мадагаскаре. Или что угодно любое другое. Связанные темы:
        \begin{enumerate}
            \item Экологические проблемы в вашем родном регионе — берите именно ваш родной регион!
        \end{enumerate}
        \item Outstanding russian scientist — мы составляли презентации, из них надо сделать 25 фраз и выучить.
    \end{enumerate}
\end{enumerate}

\subsection{Домашнее задание}

\begin{enumerate}
    \item Найти информацию о недавно (в течение прошедших 0.5 лет) прошедшей конференции (научной) по нашей специальности. Где, когда, что обсуждали, к каким выводам пришли?
    \item Составить и быть готовым рассказывать наизусть 25 фраз по теме «The MTUCI»
    \item Прочитать и перевести текст «Conferences, symposiums, congresses» (p. 184)
\end{enumerate}

\end{document}