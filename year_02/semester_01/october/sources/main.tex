\documentclass{article}
\usepackage[utf8]{inputenc}

\usepackage[T2A]{fontenc}
\usepackage[utf8]{inputenc}
\usepackage[russian]{babel}

\usepackage{tabularx}
\usepackage{amsmath}
\usepackage{pgfplots}
\usepackage{geometry}
\usepackage{multicol}

\geometry{
    left=1cm,right=1cm,top=2cm,bottom=2cm
}
\newcommand*\diff{\mathop{}\!\mathrm{d}}

\newtheorem{definition}{Определение}
\newtheorem{theorem}{Теорема}

\DeclareMathOperator{\sign}{sign}

\usepackage{hyperref}
\hypersetup{
    colorlinks, citecolor=black, filecolor=black, linkcolor=black, urlcolor=black
}

\title{Английский язык}
\author{Лисид Лаконский}
\date{October 2023}

\begin{document}
\raggedright

\maketitle

\tableofcontents
\pagebreak

\section{Практическое занятие — 14.10.2023}

\subsection{Reading 1}

\subsubsection{Exercise 1}

\textbf{Dicuss the following with your partner and choose the best answer:}

People talk a lot in the information technology business about "intellectual property rights". What are they?

\begin{enumerate}
    \item How do they apply to software technology?
    
    Intellectual property rights protect also software technologies.
    \item Why should you protect them?
    
    To protect my assets from theft.
    \item How do you protect them?
    
    I protect my software technologies with licenses.
\end{enumerate}

\subsubsection{Exercise 2}

\textbf{Match the words (1-10) with the definitions (A-J)}:

\begin{multicols}{2}
\begin{enumerate}
    \item patent — limited legal monopoly...
    \item copyright — legal monopoly that protects...
    \item digital rights — technology through which the owner...
    \item infringement — violation...
    \item plagiarism — stealing of words... 
    \item software license — permission to use a software...
    \item asset — something valuable...
    \item idea generation — the process of creating...
    \item innovation — the process of translating...
    \item ownership — the ultimate and exclusive right...
\end{enumerate}
\end{multicols}

\subsubsection{Exercise 3}

\textbf{Answer the following questions:}

\begin{enumerate}
    \item What are intellectual property (IP) rights?
    
    Intellectual property rights are rights of creator of some invention, design or something else to prevent others from using it, to negotiate payment in return for others using them. 
    \item What forms of IP do you know?
    
    There are four types of intellectual property rights: patents, copyrights, trade secret and trademarks.
    \item How do IP rights apply in software technology?
    
    Patents, copyrights, trade secrets protect the technology itself, trademarks protect the names or symbols used to distinguish a product in the marketplace.
    \item What can a patent protect?
    
    Patent can protect ideas, systems, methods, algorithms, and functions embodied in a software product: editing functions, user-interface features, compiling techniques, etc.
    \item What can copyrights protect?
    
    In the case of software, copyright law would protect the source and object code, as well as certain unique original elements of the user interface.
    \item Why do companies have trade secrets?

    Companies have trade secrets to have competitive advantage.
    \item What features of software can be protected as trade secrets?
    
    Code and the ideas and concepts reflected in it can be protected as trade secrets.
\end{enumerate}

\subsubsection{Exercise 4}

\textbf{Match paragraph (1–7) with the correct heading (A-G)}

\begin{multicols}{2}
\begin{enumerate}
    \item Paragraph 1 — Ideas and knowledge
    \item Paragraph 2 — Intellectual property rights
    \item Paragraph 3 — Patent
    \item Paragraph 4 — Software patents
    \item Paragraph 5 — Copyrights protection
    \item Paragraph 6 — Exclusive right of a copyright owner
    \item Paragraph 7 — Trade secrets
\end{enumerate}
\end{multicols}

\subsubsection{Exercise 5}

\textbf{Read the text againi and decide whether these stateents are true, false, or information is not available}

\begin{enumerate}
    \item 
\end{enumerate}

\subsection{Домашнее задание}

\begin{enumerate}
    \item Подготовить презентацию на 5-7 минут про какого-либо russian scientist.
\end{enumerate}

\section{Практическое занятие — 28.10.2023}

\subsection{Intellectual Property: Software Protection, Resume}

\paragraph{Introduction}

\textbf{The article is about} intellectual property rights in the context of software.

\textbf{The author talks about} four types of intellectual property rights relevant to software: patents, copyrights, trade secrets and trademarks, the areas of applicability of each of them and the differences beetween them.

\paragraph{Main Content}

\textbf{According to the article}, a patent is a twenty years exclusive monopoly on the right to make, use and sell an invention which can protect features of a program that cannot be protected by other means. For example, ideas, methods, algorithms.

\textbf{Next, the author talks about a copyright and how it differs from a patent}. While a patent can protect the novel ideas, a copyright cannot. Copyright is intended to protect only the original expression of an idea. In the case of software, copyright law could protect the source and object code. Unlike a patent, copyright protection arises automatically upon the creation of a creative work.

\textbf{Finally, the author talks about trade secrets and trademarks}. A trade secret is an intellectual property that has inherent economic value because it is not known by others, and which the owner takes measures to keep secret. When it comes to software, the code and ideas expressed in it may be protected as trade secrets. This protection lasts as long as the protected element retains its trade secret status.

\textbf{Unlike the other intellectual property rights}, trademarks do not protect product, but the names or symbols that identifies product from a particular source and distinguishes it from others.

\paragraph{Conclusion}

\textbf{The author concludes} that in order to protect and obtain maximum benefit from a software asset, it is critical to have a good understanding of the intellectual property rights and how best to use the available forms of legal protection to protect those rights.

\end{document}